\chapter{Conclusion}
Hoofdstuk 5 Conclusie / discussie: de conclusie moet aansluiten op de inleiding. De lezer die de inleiding en de conclusie leest moet een helder beeld van het onderzoek /experiment gekregen hebben. In de conclusie worden antwoorden op de onderzoeksvragen en de resultaten van het experiment vergeleken met de literatuurwaarden of met je eigen verwachting. Werk in een conclusie altijd met getallen. Woorden als redelijk, matig, vrij goed, niet goed, enz. mogen niet erin voorkomen. Er zijn twee mogelijkheden: het resultaat van het experiment en de literatuurwaarde stemmen overeen of er is een (systematische) afwijking tussen het resultaat van het experiment en de literatuurwaarde. Zoek hiervoor een verklaring en geef commentaar. Aanbevelingen voor een vervolgonderzoek kunnen ook opgenomen worden. Geef geen nieuwe informatie in de conclusie en geen verwijzingen opnemen.