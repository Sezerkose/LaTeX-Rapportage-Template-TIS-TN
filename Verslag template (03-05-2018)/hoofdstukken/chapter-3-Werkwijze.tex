\chapter{Operating Procedure (Voorkeur naar eigen informatieve titel)}

Hoofdstuk 3 Werkwijze: geef in dit hoofdstuk een beschrijving van de opstelling, de onderdelen plus de karakteristieke eigenschappen van de opstelling. Maak een schematisch overzicht van de opstelling (geen foto!) en stel jezelf de vraag of de lezer het experiment op basis van de door jou verschafte informatie op precies dezelfde manier kan herhalen.

Leg bij de meetprocedure uit:
- hoe er is gemeten is a.d.h.v. een schematische voorstelling van de meetopstelling
- welke (apparatuur) instellingen er gebruikt zijn
- welke systematische fout aanwezig is en hoe je dit compenseert
- hoe vaak een meting herhaald is
- hoe lang er gemeten is
- Indien van toepassing wordt ook de bereiding van het te nemen monster (sample bereiding) beschreven en het aantal monsters.