\newpage
Bijlagen: in de bijlagen komen bijvoorbeeld series grafieken, (grote) tabellen, omvangrijke afleidingen van formules, kopieën van gebruikte documentatie en andere informatie. In het algemeen kan gezegd worden: informatie, welke door hun omvang of indirecte belang de leesbaarheid van het verslag verstoren, wordt opgenomen in de bijlagen. Bij grote hoeveelheid gegevens (data) kan gebruik gemaakt worden van een aparte bijlage (los van het verslag) of een elektronische informatiedrager. Bijlagen worden voorzien van volgnummer en informatieve titel.

\chapter{Measurements}
Tabellen of github link naar de meetwaarden. Eventueel figuren.

\chapter{\textit{Meetplan}}
Bevat het wekelijkse meetplan, indien hiernaar verwijst gaat worden in het verslag.