\chapter{Discussion}
In Figure \ref{fig:silver1-6} a bulge in the Hall voltage with respect to the fit between 6-10 A is visible. This is due to the fact that two different Hall apparatuses have been used. One of which (58681) had a rapidly increasing offset voltage, due to which its temperature increased a lot more than the other used apparatus (58681 B2). Any measurement with this apparatus after 10 A was unsuccessful. This is due to the fact that the voltage kept increasing for about 30 minutes per ampere. Due to this constant increase in the offset voltage, this bulge (6-10 A) has a greater uncertainty. As can be seen in Figure \ref{fig:silver1-6} the uncertainties in the negative direction of the Hall voltage of these points overlap the fit.\\
This experiment has not been without any trouble, as could be expected while measuring voltages in the order of 10$^{-6}$ without any special equipment in an area with electromagnetic noise. At first there was no multimeter available which supported $\mu$V precision. Therefor an op-amp circuit was required which amplified the signal 500 times. The signal was decent, however noise was visibly present while touching or moving anything from the experimental setup by just a tiny bit. Therefor great care was required to get successful and precise measurements, nothing has to be touched or moved. \\
Another problem was that the first results did not agree with the literature values. After supplying the coils with 15 A instead of 5 A of current it was known why. Since the coils had to be turned off in-between individual measurements, it could be seen that the offset voltage increased with the increase in current through the Hall apparatus. With this knowledge the actual Hall voltage could be determined. \\